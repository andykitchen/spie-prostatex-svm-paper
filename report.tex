\documentclass[a4paper]{spie}

\renewcommand{\baselinestretch}{1.0} % Change to 1.65 for double spacing
 
\usepackage{amsmath,amsfonts,amssymb}
\usepackage{graphicx}
\usepackage[colorlinks=true, allcolors=blue]{hyperref}

\title{Support Vector Machines for Prostate Lesion Classification in the ProstateX Challenge}

\author[a]{Andy Kitchen}
\author[b]{Jarrel Seah}

\affil[a]{Silverpond,
  Level 2 / 382 Little Collins Street,
  Melbourne, Australia}
\affil[b]{Alfred Hospital,
  55 Commercial Road,
  Melbourne, Australia}

\authorinfo{Further author information:\\
  Andy Kitchen: E-mail: andy.kitchen@silverpond.com.au, Telephone: +61 432514287\\
  Jarrel Seah: E-mail: J.Scy@alfred.org.au, Telephone: +61 450 681 551}

% Option to view page numbers
\pagestyle{empty} % change to \pagestyle{plain} for page numbers   
\setcounter{page}{1} % Set start page numbering at e.g. 301

\begin{document} 
\maketitle

\begin{abstract}
Support vector machines (SVM) are applied to the problem of prostate lesion classification for the SPIE ProstateX Challenge 2016, achieving a score of 0.82 area under the curve (AUC) on held out test data. Square 5mm transverse image patches are extracted around each lesion center from multiple aligned magnetic resonance imaging (MRI) scans. Three MRI modalities are simultaneously analyized: T2-weighted, apparent diffusion coefficient (ADC) and volume transfer constant ($K^{trans}$). Extracted patches are used to train a binary classifier to predict ``clinical significance''.  The machine learning algorithm is trained on 76 positive cases and 254 negative cases (330 total) from the challenge. The method is conceptually simple, trains in a few seconds and yields competitive results.
\end{abstract}

\section{INTRODUCTION}

The Support Vector Machine\cite{CortesVapnik1995} (SVM) is applied to the problem of prostate lesion classification for the SPIE ProstateX Challenge 2016. This machine learning algorithm is trained on magnetic resonance imaging (MRI) scans to classify new unseen lesions as clinically significant or not. The method described achives competitive performance with a score of 0.82 area under the curve (AUC) assessed on held out test data with correct answers kept hidden by competition organisers. This approach is conceptually simple. There are no involved processing steps or complex computer vision algorithms. The method is efficient, due to its simplicity and the availability of highly optimized implementations of SVMs. The implementation uses free and open source software tools and libraries, so there are no encumberances to further research or reproduction; all code is extensible and auditable.

The data provided by the challenge is a collection of patients each with MRI images in multiple modalities and metadata provided as comma separated value (CSV) files. For each patient one or more prostate lesions and their locations have been identified. For patients in the training set, each lesion is labeled with its clinical significance (true or false). The task is to predict the hidden labels for the patients in the test set. Three MRI modalities are simultaneously analyized by this method: T2-weighted, apparent diffusion coefficient (ADC) and volume transfer constant ($K^{\mathit{trans}}$); which are all shown to be related to clincal significance\cite{langer2010prostate}. These modalities are all aligned and processed together.

\section{METHOD}

\subsection{Preprocessing}

The ProstateX input data is complex, including multiple data structures and formats; one major challenge is processing and reconciling inconsistent or overlapping metadata. Multiple rules were applied on each patient to heuristically extract the most relevent T2, ADC and $K^{\mathit{trans}}$ images. Metadata was collated from both Comma Separated Value (CSV) files and information embedded within image files themselves. Significant effort was expended in this area, rivaling that spent on machine learning models.

\subsection{Patch Extraction}

For each patient, and each lesion, a centered $5\text{mm} \times 5\text{mm}$ patch is extracted at a resolution of 1px/mm in 3 modalities T2, ADC and $K^{\mathit{trans}}$. Only transverse image slices are used. Images are flattened into a 75 ($5 \times 5 \times 3$) dimensional vector. We further augment this vector with zone information by encoding it as dummy variables\cite{Hastie2009} which is then concatenated with the image vector. See table \ref{table:performance} for performance comparison without zone information.

Patch extraction subroutines were validated using pixel-by-pixel comparison to the reference images released by the challenge organisers and were found to be in close agreement. All processing was internally carried out with 32-bit floating point pixel values. This preserves large dynamic range and subtle contrast differences important for later analysis.

\subsection{Normalization}

Each input dimension has the mean subtracted and is divided by the standard deviation. This ensures that the distribution of each dimension is approximately normally distributed. $K^{\mathit{trans}}$ values are also transformed with a log transform to correct for large skew.

\subsection{Example Weighting}

ProstateX training labels are highly unbalanced, with 76 positives and 254 negatives. It is necessary to weight the classes appropriately\cite{Yang2009}, the following formula is used:

\begin{equation}
\text{class weight} =
\frac
    {\text{total number of examples}}
    {\text{number of classes} \times \text{number of examples in class}}
\end{equation}
    
The positive classes are given a weight of $ 330 / (2 \times 76) =  2.17 $ and negative classes a weight of $ 330 / (2 \times 254) = 0.65 $. This can be interpreted as making it approximately three times worse to incorrectly label a positive case than a negative case during training.

\subsection{Kernel Selection}

The SVM kernel selected is the radial basis function (RBF); together called the RBF-SVM. This is a non-linear kernel so increases/decreases in one dimension do not necessarily cause proportional changes in score output. The authors beleve that this is a desirable property. Empirically, linear models also performed much worse and were abandoned early on.

\subsection{Hyperparameter Selection}

RBF-SVMs require tuning of two hyper-parameters a regularization parameter $C$ and a kernel parameter $\gamma$. The highest scoring settings achieved with 3-fold cross validation\cite{Hastie2009} are used. Where the final AUC is calculated by averaging AUC over every fold. A simple grid search is carried out where each combination of $C \in \{0.1, 0.5, 1, 2, 5, 10, 20, 30, 50\}$ and $\gamma \in \{10^{-1}, 10^{-2}, 10^{-3}, 10^{-4}, 10^{-5}\}$ is scored and the best combination selected, see table \ref{table:performance} for selected values.

\subsection{Scoring}

The ProstateX challenge required entries to provide a continuous significance score for each test lesion, while SVMs produce a discrete binary classification. However a score can be easily derived for SVMs by using the decision function value directly instead of just the sign. This score increases in magnitude as the example moves further away from the SVM decision boundary. E.g. A score near zero indicates that small changes to this input would cause the prediction to change; while conversely, a large positive or negative value indicates that a small change in this input would not lead the prediction to change. This derived score is practically effective for this task.

\subsection{Performance Comparison}

A performance comparison between different model configuations is presented in table \ref{table:performance}.

\begin{table}[h]
  \caption{Comparison of cross validation performance for differing configurations}
  \label{table:performance}

  \begin{center}
  \begin{tabular}{ l | r | r | r }
    & AUC  & C  & $\gamma$ \\
    \hline
    Image patch only    & .782 & 50 & $10^{-5}$  \\
    With zone           & .806 & 30 & $10^{-5}$  \\
    With class weights  & .811 & 30 & $10^{-3}$  \\
  \end{tabular}
  \end{center}
\end{table}

\section{Implementation}

This competition entry is reproducable and implemented using only open source software including Python, PyDICOM, SimpleITK\cite{simpleitk}, Scikit-Learn\cite{scikit-learn} and NumPy\cite{numpy}. Source code is available from the authors under an open source license.

% References
\bibliography{report} % bibliography data in report.bib
\bibliographystyle{spiebib} % makes bibtex use spiebib.bst

\end{document}
